\documentclass{beamer}
\usepackage{amsfonts}
\usepackage[MeX]{polski}
\usepackage[utf8]{inputenc}



\title{Prezentacja}
\author{Oskar Polny}
\date{09 listopada 2017}
\begin{document}

\frame{\titlepage}
\begin{frame}
\frametitle{Spis treści}
\tableofcontents
\end{frame}
\section{Intel}
\begin{frame}{Definicja}
Intel – największy na świecie producent układów scalonych oraz twórca mikroprocesorów z rodziny x86, które znajdują się w większości komputerów osobistych.
\begin{figure}
\centering
\includegraphics[scale=0.20]{Intel.jpg}
\end{figure}
\end{frame}
\begin{frame}{Informacje}
\begin{itemize}
\item<1-2> Oprócz mikroprocesorów wytwarza między innymi płyty gółwne, chipsety do płyt głównych, zintegrowane układy graficzne, pamięci Flash, mikrokontrolery, procesory do systemów wbudowanych, sprzęt sieciowy, systemy zarządzania pamięcią masową
\pause
\item<-2> Od sezonu 2014/2015 jest oficjalnym sponsorem katalońskiego klubu piłkarskiego FC Barcelona..
\end{itemize}
\end{frame}
\begin{frame}{Wersje procesorów}
PrzykĹ‚adowe Wersje procesorów:
\begin{itemize}
\item<1-5> Pentium – wersje M (do laptopów), wersje podstawowe 2,3,4
\item<2-5> Celeron – wersje M (do laptopów) i D, Xeon i Itanium – procesory do serwerów
\item<3-5> Core oraz Core 2 – procesory jedno-, dwu- i czterordzeniowe.
\item<4-5>  Core i9 cztero-, sześcio-, dwunasto-, i osiemnastordzeniowe
\item<5>  Core i7 sześcio- i czterordzeniowe oraz cztero- i dwurdzeniowe Core i5 i dwurdzeniowe Core i3
\end{itemize}
\end{frame}
\end{document}